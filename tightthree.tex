\documentclass[11pt]{article}

\usepackage{tikz}

\usepackage{amsmath,amsfonts,amsthm}
\usepackage{algorithm,algorithmicx}
\usepackage{algpseudocode}
\usepackage{enumerate}
%\usetikzlibrary{calc}
\textheight9in \textwidth6in

\hoffset-0.55in \voffset-0.6in

%\pagestyle{empty}

\newtheorem{defn}{{\bf Definition}}[section]
\newtheorem{example}[defn]{{\bf Example}} \newtheorem{lemma}[defn]{{\bf
Lemma}} \newtheorem{prop}[defn]{{\bf Proposition}}
\newtheorem{theorem}[defn]{{\bf Theorem}} \newtheorem{cor}[defn]{{\bf
Corollary}} \newtheorem{remark}[defn]{{\bf Remark}}
\newtheorem{conj}[defn]{{\bf Conjecture}} \newtheorem{qn}[defn]{{\bf
Question}}


                        %%% AMS fonts%%%


\font\bbb=msbm10 scaled\magstep1

\newcommand{\BB}{\mbox{\bbb B}} \newcommand{\CC}{\mbox{\bbb C}}
\newcommand{\FF}{\mbox{\bbb F}} \newcommand{\HH}{\mbox{\bbb H}}
\newcommand{\NN}{\mbox{\bbb N}} \newcommand{\OO}{\mbox{\bbb O}}
\newcommand{\QQ}{\mbox{\bbb Q}} \newcommand{\RR}{\mbox{\bbb R}}
\newcommand{\ZZ}{\mbox{\bbb Z}} \newcommand{\N}{\mbox{I\!\!\!N}}

\font\bbb=msbm8 scaled\magstep1

\newcommand{\BBB}{\mbox{\bbb B}} \newcommand{\CCC}{\mbox{\bbb C}}
\newcommand{\FFF}{\mbox{\bbb F}} \newcommand{\HHH}{\mbox{\bbb H}}
\newcommand{\NNN}{\mbox{\bbb N}} \newcommand{\OOO}{\mbox{\bbb O}}
\newcommand{\QQQ}{\mbox{\bbb Q}} \newcommand{\RRR}{\mbox{\bbb R}}
\newcommand{\ZZZ}{\mbox{\bbb Z}}


\def\scoll{\mbox{{\scriptsize $\,\searrow$}$\!\!\stackrel{{}^{\rm
s}}{}\,\,$}}

\def\coll{\mbox{\scriptsize $\,\,\searrow\,$}}

\newcommand{\bdN}{N^{\!\!\!^{^{\bullet}}}}
\newcommand{\TPSS}{S^{\hspace{.2mm}3} \mbox{$\times
\hspace{-2.8mm}_{-}$} \, S^{\hspace{.1mm}1}}
\newcommand{\TPSSS}{S^{\hspace{.2mm}3} \mbox{$\times
\hspace{-4.5mm}_{-}$} \, S^{\hspace{.1mm}1}}
\newcommand{\TPSSD}{S^{\hspace{.2mm}d-1} \mbox{$\times
\hspace{-2.8mm}_{-}$} \, S^{\hspace{.1mm}1}}
\newcommand{\TPSSDS}{S^{\hspace{.2mm}d-1} \mbox{$\times
\hspace{-3.3mm}_{-}$} \, S^{\hspace{.1mm}1}}
\newcommand{\TPSSMS}{S^{\hspace{.2mm}m} \mbox{$\times
\hspace{-3.3mm}_{-}$} \, S^{\hspace{.1mm}1}}
\newcommand{\TPSST}{S^{\hspace{.2mm}2} \mbox{$\times
\hspace{-2.8mm}_{-}$} \, S^{\hspace{.1mm}1}}

\newcommand{\XB}{X \hspace{-2.7mm}^{^{\mbox{\bf --}}}}



 \newcommand{\TPPSS}{\kern.24em \rule width.08em height1.5ex
depth-.08ex \kern-.36em \times}


\def\smlbox{\mbox{ \setlength{\unitlength}{1mm}
\begin{picture}(4.3,0)(0,0) \put(0,2){\mbox{\tiny ${}_{{}^{
\begin{array}[b]{|c|} \hline \\[-1.7mm] \hspace{-1.4mm}{}^{\bullet}
\hspace{-0.6mm} \angle {}^{\bullet}\hspace{-1.6mm}\\[-1mm]
\hspace{-1.4mm}{\large \smile}\hspace{-1.6mm}\\[-.2mm] \hline
\end{array}}} $ }} \end{picture} }}

\font\bb=msbm10 scaled\magstep2 \newcommand{\C}{\mbox{\bb C}}


\newcommand{\al}{\alpha} \newcommand{\ba}{\beta}
\newcommand{\ga}{\gamma} \renewcommand{\th}{\theta}
\newcommand{\Om}{\Omega} \newcommand{\La}{\Lambda}
\newcommand{\noi}{\noindent} \newcommand{\si}{\sigma}
\newcommand{\Ga}{\Gamma} \newcommand{\De}{\Delta}
\newcommand{\Lk}[2]{{\rm lk}_{#1}(#2)} \newcommand{\st}[2]{{\rm
st}_{#1}(#2)}
\newcommand{\mb}{\mathbb}
\newcommand{\tr}[1]{{\cal D}T_#1}
\newcommand{\ovr}[1]{\overrightarrow{#1}}
\newcommand{\trt}[1]{treetype(#1)}

\begin{document}

%\setcounter{page}{97}

\title{\bf Tight triangulations of three manifolds}

%\date{}

\maketitle

\vspace{-5mm}





\begin{center}
\date{}
\end{center}





\hrule

\bigskip

\noindent {\bf Abstract.}

\smallskip
{\small Tight triangulations are exotic objects in combinatorial topology.
Any piecewise linear embedding of a tight triangulation into euclidean
space is as convex as allowed by the topology of the underlying space.
Tight triangulations are also conjectured to be minimal, and proven to be
so for dimensions two and three. Inspite of substantial theoretical results
about such triangulations, there are precious few examples. Infact, apart
from dimension two, we do not know if there are infinitely many of them in
any dimension. In this paper, we present a computer friendly combinatorial
scheme to obtain several three dimensional tight triangulations. While we
still do not know if there are infinitely many of them, it does look like
there are abundantly many, if we look for them the right way.}

\bigskip

\noindent {\small {\em MSC 2010\,:} 57Q15, 57R05.

\noindent {\em Keywords:} Stacked sphere; Tight triangulation;
Strongly minimal triangulation.

}

\bigskip

\hrule

\section{Introduction}
In this paper we present some tight triangulations of some three manifolds.
We prove an adaptation of the construction method used in \cite{bdns},
which is more amenable to computer processing. The triangulations we obtain
are 
neighborly members of Walkup's class ${\cal K}(d)$, with each vertex-link as a
stacked sphere. All our triangulations triangulate connected sums of
$1$-handlebodies. Among the known constructions of such manifolds are:
the unique 9-vertex triangulation of $(\TPSST)$ of K\"uhnel, apart from the
two three dimensional members
of the infinite family constructed in \cite{bdns}.



\section{Preliminaries}


All simplicial complexes considered here are finite and abstract.
By a triangulated manifold/sphere/ball, we mean an abstract
simplicial complex whose geometric carrier is a topological
manifold/sphere/ball. We identify two complexes if they are
isomorphic. A $d$-dimensional simplicial complex is called {\em
pure} if all its maximal faces (called {\em facets}) are
$d$-dimensional. A $d$-dimensional pure simplicial complex is said
to be a {\em weak pseudomanifold} if each of its $(d - 1)$-faces
is in at most two facets. For a $d$-dimensional weak
pseudomanifold $X$, the {\em boundary} $\partial X$ of $X$ is the
pure subcomplex of $X$ whose facets are those $(d-1)$-dimensional
faces of $X$ which are contained in unique facets of $X$. The {\em
dual graph} $\Lambda(X)$ of a pure simplicial complex $X$ is the
graph whose vertices are the facets of $X$, where two facets are
adjacent in $\Lambda(X)$ if they intersect in a face of
codimension one. A {\em pseudomanifold} is a weak pseudomanifold
with a connected dual graph. All connected triangulated manifolds
are automatically pseudomanifolds.


If $X$ is a $d$-dimensional simplicial complex then, for $0\leq j
\leq d$, the number of its $j$-faces is denoted by $f_j = f_j(X)$.
The vector $f(X) := (f_0, \dots, f_d)$ is called the {\em face
vector} of $X$ and the number $\chi(X) := \sum_{i=0}^{d} (-1)^i
f_i$ is called the {\em Euler characteristic} of $X$. As is well
known, $\chi(X)$ is a topological invariant, i.e., it depends only
on the homeomorphic type of $|X|$. A simplicial complex $X$ is
said to be {\em $l$-neighborly} if any $l$ vertices of $X$ form a
face of $X$. A 2-neighborly simplicial complex is also called a
{\em neighborly} simplicial complex.

A {\em standard $d$-ball} is a pure $d$-dimensional simplicial
complex with one facet. The standard ball with facet $\sigma$ is
denoted by $\overline{\sigma}$. A $d$-dimensional pure simplicial
complex $X$ is called a {\em stacked $d$-ball} if there exists a
sequence $B_1, \dots, B_m$ of pure simplicial complexes such that
$B_1$ is a standard $d$-ball, $B_m=X$ and, for $2\leq i\leq m$,
$B_i = B_{i-1}\cup \overline{\sigma_i}$ and $B_{i-1} \cap
\overline{\sigma_i} = \overline{\tau_i}$, where $\sigma_i$ is a
$d$-face and $\tau_i$ is a $(d-1)$-face of $\sigma_i$. Clearly, a
stacked ball is a pseudomanifold. A simplicial complex is called a
{\em stacked $d$-sphere} if it is the boundary of a stacked
$(d+1)$-ball. A trivial induction on $m$ shows that a stacked
$d$-ball actually triangulates a topological $d$-ball, and hence a
stacked $d$-sphere is a triangulated $d$-sphere. If $X$ is a
stacked ball then clearly $\Lambda(X)$ is a tree. So, a stacked ball
is a pseudomanifold whose dual graph is a tree. But, the converse
is not true (e.g., the 3-pseudomanifold $X$ whose facets are
$1234, 2345, 3456, 4567, 5671$ is a pseudomanifold for which
$\Lambda(X)$ is a tree but $|X|$ is not a ball). The following results appear
in \cite{bdns}.

\begin{prop}\label{prop:stackedball} Let $X$ be a pure
$d$-dimensional simplicial complex.  \begin{enumerate} \item[{\rm
(i)}] If $\Lambda(X)$ is a tree then $f_0(X) \leq f_d(X) +d$.
\item[{\rm (ii)}] $\Lambda(X)$ is a tree and $f_0(X) = f_d(X) +d$
if and only if $X$ is a stacked ball.  \end{enumerate}
\end{prop}

\begin{cor}\label{cor:C0} Let $X$ be a pure $d$-dimensional
simplicial complex and let $CX$ denote a {\em cone} over $X$.
Then $CX$ is a stacked $(d+1)$-ball if and only if $X$ is a
stacked $d$-ball.
\end{cor}

In \cite{wa}, Walkup defined the class ${\cal K}(d)$ as the family
of all $d$-dimensional simplicial complexes all whose vertex-links
are stacked $(d - 1)$-spheres. Clearly, all the members of ${\cal
K}(d)$ are triangulated closed manifolds. Let ${\cal K}^{\ast}(d)$
be the class of 2-neighborly members of  ${\cal K}(d)$. We know
the following.

\begin{prop}[Bagchi and Datta \cite{bd16}]\label{P2}
Let $M$ be a connected closed triangulated manifold of dimension
$d\geq 3$. Let $\beta_1=\beta_1(M;\ZZ_{2})$. Then the face vector
of $M$ satisfies:
\begin{enumerate}[{\rm (a)}]
\item $ f_j \geq \begin{cases}
            \binom{d+1}{j}f_0+j\binom{d+2}{j+1}(\beta_1-1), & \mbox{
if } 1\leq j<d, \\
            df_0+(d-1)(d+2)(\beta_1-1), & \mbox{ if } j=d.
        \end{cases}
$

\item $\binom{f_0-d-1}{2}\geq \binom{d+2}{2}\beta_1$.
\end{enumerate}
When $d\geq 4$, the equality holds in {\rm (a)} $($for some $j
\geq 1$, equivalently, for all $j\,)$ if and only if $M\in {\cal
K}(d)$, and equality holds in {\rm (b)} if and only if $M\in
{\cal K}^{\ast}(d)$.
\end{prop}

The case $d=4$ of the above proposition is due to Walkup \cite{wa}
and K\"{u}hnel \cite{ku}. Part (b) of the above proposition is due
to Novik and Swartz \cite{NS}.

\begin{prop}[Kalai \cite{ka}]\label{P3} For $d\geq 4$, a connected
simplicial complex $X$ is in ${\cal K}(d)$ if and only if $X$ is
obtained from a stacked $d$-sphere by $\beta_1(X)$ combinatorial
handle additions. In consequence, any such $X$ triangulates either
$(S^{\,d -1}\!\times S^1)^{\# \beta_1}$ or $(\TPSSD)^{\#\beta_1}$
according as $X$ is orientable or not. $($Here $\beta_1 =
\beta_1(X).)$ \end{prop}

It follows from Proposition \ref{P3}  that \begin{eqnarray}
\label{eq:beta1} \chi(X) = 2 - 2 \beta_1(X) \, \mbox{ for } \,
X\in {\cal K}(d).  \end{eqnarray}

For a field $\FF$, a $d$-dimensional simplicial complex $X$ is
called {\em tight with respect to} $\FF$ (or {\em $\FF$-tight}) if
(i) $X$ is connected, and (ii) for all induced subcomplexes $Y$ of
$X$ and for all $0\leq j \leq d$, the morphism $H_{j}(Y; \FF) \to
H_{j}(X; \FF)$ induced by the inclusion map $Y \hookrightarrow X$
is injective. If $X$ is $\QQ$-tight then it is $\FF$-tight for all
fields $\FF$ and called {\em tight} (cf. \cite{bd17}).

A $d$-dimensional simplicial complex $X$ is called {\em minimal}
if $f_0(X) \leq f_0(Y)$ for every triangulation $Y$ of the
geometric carrier $|X|$ of $X$. We say that $X$ is {\em strongly
minimal} if $f_i(X) \leq f_i(Y)$, $0\leq i \leq d$, for all such
$Y$. We know the following.


\begin{prop}[Effenberger \cite{ef}, Bagchi and Datta
\cite{bd16}]\label{P4} Every $\FF$-orientable member of ${\cal
K}^{\ast}(d)$ is $\FF$-tight for $d\neq 3$. An $\FF$-orientable
member of ${\cal K}^{\ast}(3)$ is $\FF$-tight if and only if
$\beta_1(X)=(f_0(X)-4)(f_0(X)-5)/20$.
\end{prop}

\begin{prop}[Bagchi and Datta \cite{bd16}]\label{P5} Every
$\FF$-tight member of\,  ${\cal K}(d)$ is strongly minimal.
\end{prop}

Let $\overline{{\cal K}}(d)$ be the class of all $d$-dimensional
simplicial complexes all whose vertex-links are stacked
$(d-1)$-balls. Clearly, if $N \in \overline{{\cal K}}(d)$ then $N$
is a triangulated manifold with boundary and satisfies
\begin{eqnarray} \label{eq:skel} {\rm skel}_{d-2}(N) = {\rm
skel}_{d-2}(\partial N).  \end{eqnarray} Here ${\rm skel}_{j}(N) =
\{\alpha\in N \, : \, \dim(\alpha) \leq j\}$ is the $j$-skeleton
of $N$. We know the following.

\begin{prop}[Bagchi and Datta \cite{bd17}]\label{P6}
For $d \geq 4$, $M\mapsto \partial M$ is a bijection from
$\overline{\cal K}(d+1)$ to ${\cal K}(d)$.
\end{prop}

From the above proposition, we have the following:
\begin{cor}\label{cor:C1}
For $d\geq 4$, if $M\in \overline{\cal K}(d+1)$ then ${\rm
Aut}(M)={\rm Aut}(\partial M)$.
\end{cor}

Let $G$ be a graph and ${\cal T}=\{T_i\}_{i\in {\cal I}}$ be a family of induced subtrees of $G$, such that every vertex of $G$ is contained in exactly $d+1$ trees and any two adjacent vertices appear together in exactly $d$ trees. We define the pure $d$-dimensional simplicial complex ${\cal K}(G,{\cal T})$ by the following facet complex:
\begin{equation}\label{eq:scdual}
{\cal K}(G,{\cal T}) := \{\{i: u\in T_i\}: u\in V(G)\}.
\end{equation}
We will denote the facet $\{i: u\in T_i\}$ by $\hat{u}$ for $u\in V(G)$.
Our constructions are based on following result from \cite{bdns}.

\begin{prop}\label{prop:construction}
Let $G$ be a graph and ${\cal T} = \{T_i\}_{i=1}^{n}$ be a  family
of $(n-d)$-vertex induced subtrees of $G$, any two of which
intersect. Suppose that {\rm (i)} each vertex of $G$ is in exactly
$d+1$ members of $\cal T$  and {\rm (ii)} for any two vertices
$u\neq v$ of $G$, $u$ and $v$ are together in exactly $d$ members
of $\cal T$ if and only if $uv$ is an edge of $G$. Then ${\cal K}(G,{\cal T})$ is a neighborly member of $\overline{\cal K}(d)$, with
$\Lambda({\cal K}(G,{\cal T}))\cong G$.
\end{prop}

\section{Examples of tight three manifolds}
In this section we present examples of tight three dimensional manifolds
obtained as boundaries of tight four manifolds with boundary. More explicitly
for each $n=20k+9,2\leq k\leq 5$, we construct $n$-vertex triangulations using
Proposition 
\ref{prop:construction} which have $\mb{Z}_n$ as the automorphism group. We
need some notation to describe our construction.

Let $k\geq 1$, $n = 20k+9$ and $d_0,d_1,\ldots, d_k$ be invertible elements
in $(\mathbb{Z}_n,\cdot)$.  Let $G := G(k;d_0,\ldots,d_k)$ denote the graph
on $n(4k+1)$ vertices with the vertex set $V(G)$ given by 
$V(G) := \{v_{i,j}: 0\leq i\leq 4k, j\in \mb{Z}_n\}$. The edge set $E(G)$ of
$G$ consists of edges $(v_{i,j},v_{i+1,j})$ for $0\leq i<4k, j\in \mb{Z}_n$,
and edges $(v_{4i,j},v_{4i,j+d_i})$ for $0\leq i\leq k$ and $j\in
\mathbb{Z}_n$ (Note that the second
subscript is read modulo $n$). In the graph $G(k;d_0,\ldots,d_k)$ let $P_j$
denote the path $v_{0,j}v_{1,j}\cdots v_{4k,j}$ for $j\in \mb{Z}_n$. Also let
$C_i$ denote the subgraph spanned by the edges $\{(v_{4i,j},v_{4i,j+d_i}):
j\in \mathbb{Z}_n\}$
for $0\leq i\leq k$. Note that when $d_0,\ldots,d_k$ are relatively prime to
$n$, each of the subgraphs $C_i$ is an $n$-cycle. We also note the following
automorphism of the graph $G(k;d_0,\ldots,d_k)$:
\begin{equation}\label{eq:auto}
\varphi := \prod_{i=0}^{4k} (v_{i,0},v_{i,1},\ldots,v_{i,n-1}).
\end{equation}
The automorphism $\varphi$ generates the automorphism group of $G$
isomorphic to $\mathbb{Z}_n$. To construct neighborly members of
$\overline{\cal K}(4)$, we exhibit a family of subtrees of $G$ which
satisfy the conditions in Proposition \ref{prop:construction}. 

\subsection{Construction of induced subtrees of $G$}
To describe the subtrees of $G$, we introduce some terminology. A
collection ${\cal D}=\{(\sigma_i,\tau_i):1\leq i\leq k\}$ where
$\sigma_i,\tau_i$ are permutations of the set $\{0,1,2,3\}$ is called
a $k$-{\em deck} of permutations. For a vertex $v_{i,j}\in V(G)$, we call the
subpath $v_{i-l,j}P_jv_{i,j}$ of $P_j$ to be the {\em upward} path of length $l$ at
$v_{i,j}$. Similarly we call the subpath $v_{i,j}P_jv_{i+l,j}$ of $P_j$ to be the
{\em downward} path of length $l$ at $v_{i,j}$. 

\begin{defn}\label{defn:tree}{\rm 
For a $k$-deck ${\cal D} =
\{(\sigma_i,\tau_i)\}_{i=1}^k$, let $\tr{j}$ be the induced subgraph of
$G$ spanned by the following:
\begin{enumerate}[{\rm (i)}]
\item The paths $v_{4i,j}v_{4i,j+d_i}\cdots v_{4i,j+4\cdot d_i}$ for $0\leq i\leq
k$.
\item The path $P_j$.
\item Upward paths of length $\tau_i(t)$ at vertex $v_{4i,j+(t+1)d_i}$ for
$1\leq i\leq k$ and $0\leq t\leq 3$. 
\item Downward paths of length $\sigma_{i+1}(t)$ at vertex $v_{4i,j+(t+1)d_i}$
for $0\leq i\leq k-1$ and $0\leq t\leq 3$.
\end{enumerate}
}
\end{defn}

We notice that $\tr{j}=\varphi^j(\tr{0})$. For a permutation pair
$(\sigma_i,\tau_i)\in {\cal D}$, we define its {\em span} $sp(\sigma_i,\tau_i)$ to be
the subset of $\mb{Z}_n$ given by $\{\pm ((p+1)d_{i-1}-(q+1)d_i): 0\leq p,q\leq
3, \sigma_i(p)+\tau_i(q)\geq 4\}$. We define the {\em span} of ${\cal D}$, denoted by 
$sp({\cal D})$ as:
\begin{equation}\label{eq:span}
sp({\cal D}) = \displaystyle \bigcup_{i=1}^k sp(\sigma_i,\tau_i) \cup \{\pm td_i: 0\leq t\leq 4,
0\leq i\leq k\}.
\end{equation}

\begin{lemma}\label{lem:tree}
If $sp({\cal D})=\mathbb{Z}_n$, then the subgraph ${\cal D}T_j$ is a tree for all $j=0,\ldots,n-1$.
\end{lemma}
\begin{proof}
First we show $sp({\cal D})=\mathbb{Z}_n$ implies that the sets in the union
in (\ref{eq:span}) are mutually
disjoint. Note that $|sp(\sigma_i,\tau_i)|\leq 6$. Then the number of
non-zero elements in $sp({\cal D})$ is at most $2\cdot 6k+2\cdot
4(k+1)=20k+9=n-1$. Thus all the sets in the union must be mutually disjoint
to acheive $sp({\cal D})=\mathbb{Z}_n$. Now we proceed to show that ${\cal
D}T_j$ is a tree for all $j=0,\ldots,n-1$. Notice that the steps (i) and
(ii) of the construction in Definition \ref{defn:tree}, yeild a tree
consisting of the path $P_j$ and arcs of length $4$ of cycles $C_i$ for
$0\leq i\leq k$. In the steps (iii) and (iv) we attach upward and downward
paths to the arcs attached in step (i). The resulting graph is clearly
connected. Since the new vertices introduced in steps (iii) and (iv) have
an edge only if belong to the same path $P_j$ for some $j$, for a cycle to
occur, two paths added in steps (iii) and (iv) must be subpaths of the
same path $P_j$ for some $j$. But then we must have
$j+(t+1)d_i=j+(t'+1)d_i$, or $j+(t+1)d_{i-1} = j+(t'+1)d_i$ for some $0\leq
t,t'\leq 3$.  The former implies $t=t'$, in which case the two paths are
the same. In the latter case, we have $(t+1)d_{i-1}=(t'+1)d_i$. But then the sets
$\{\pm td_i: 0\leq t\leq 4\}$ and $\{\pm td_{i-1}: 0\leq t\leq 4\}$ are not
disjoint, a contradiction. Thus the resulting graph is
connected, and without a cycle. Hence it is a tree. 
\end{proof}

We have the following:
\begin{lemma}\label{lem:span}
For $j\in sp({\cal D})$, the subgraphs $\tr{0}$ and $\tr{j}$ intersect.
\end{lemma}
\begin{proof}
First assume that $j=\pm td_i$ for some $0\leq t\leq 4$. For $j=td_i$,
$0\leq t\leq 4$, the vertex $v_{4i,td_i}$ is common to both $\tr{0}$ and
$\tr{j}$. For $j=-td_i$, $0\leq t\leq 4$, the vertex $v_{4i,0}$ is common
to both $\tr{0}$ and $\tr{j}$. Hence the subgraphs intersect. Next assume that
$j=\pm ((p+1)d_{i-1}-(q+1)d_i)$ where $0\leq p,q\leq 3$ and
$\sigma_i(p)+\tau_i(q)\geq 4$. If $j=(p+1)d_{i-1}-(q+1)d_i$, let
$r=j+(q+1)d_i=(p+1)d_{i-1}$. Then the subgraph $\tr{0}$ contains a downward
path of length $\sigma_i(p)$ at $v_{4(i-1),r}$, and the subgraph $\tr{j}$
contains an upward path of length $\tau_i(q)$ at $v_{4i,r}$. Since
$\sigma_i(p)+\tau_i(q)\geq 4$, the two paths intersect. Similarly, it can
be shown that $\tr{0}$ and $\tr{j}$ intersect when
$j=-(p+1)d_{i-1}+(q+1)d_i$.  
\end{proof}

\begin{lemma}\label{lem:permselect}
Let ${\cal D}=\{(\sigma_i,\tau_i)\}_{i=1}^k$ be a deck of permutations such
that:
\begin{enumerate}[{\rm (a)}]
\item $sp({\cal D}) = \mb{Z}_n$.
\item $\sigma_{i}(t)+\tau_{i-1}(t)\geq 1$ for $2\leq i\leq k, 0\leq t\leq 2$. 
\item $\sigma_{i}(3),\tau_{i}(3) \geq 1$ for $1\leq i\leq k$.
\end{enumerate}
Then ${\cal K}(G,{\cal D}T)$ is a neighborly member of $\overline{\cal K}(4)$ where  ${\cal D}T = \{\tr{j}\}_{j=0}^{n-1}$.
\end{lemma}

To prove Lemma \ref{lem:permselect}, we will first show an equivalent of
Proposition \ref{prop:construction}.

\begin{lemma}\label{lem:construction2}
Let $G$ be a graph and ${\cal T}=\{T_i\}_{i=1}^n$ be a family of $(n-d)$-vertex
induced subtrees of $G$, any two of which intersect. Suppose that {\rm (i)} each
vertex of $G$ is in exactly $d+1$ members of ${\cal T}$, {\rm (ii)} any two adjacent
vertices $u$ and $v$ occur together in exactly $d$ members of ${\cal T}$ and
{\rm (iii)} for a vertex $u\in V(T), T\in {\cal T}$, we have $d_G(u)-d_T(u)\leq 1$.
Then ${\cal K}(G, {\cal T})$ is a
neighborly member of $\overline{\cal K}(d)$ with $\Lambda({\cal K}(G, {\cal T}))\cong G$. 
\end{lemma}
\begin{proof}
Let $T_i\in {\cal T}$ be a tree. For a vertex $r\in T_i$, define the oriented tree
$T_i(r)$ with directed edges $\ovr{uv}$ where $uv\in T_i$ and $v$ is closer
to $r$ than $u$. Define {\em label} $l(\ovr{uv})$ to be the unique element of
$\hat{u}\backslash \hat{v}$ (follows from conditions (i) and (ii)). We prove
that all edges in $T_i(r)$ have distinct labels. Now, there are $d$ other trees
that intersect $T_i$ in $r$. Let $T_j\in {\cal T}$ be a tree that does not
intersect $T_i$ in $r$. Since any two trees in ${\cal T}$ intersect, there is a
vertex $w\neq r$ which is common to $T_i$ and $T_j$. Then we see that one of the
edges in the $w$-$r$ path in $T_i(r)$ must have the label $j$. Since there are
$n-d-1$ such trees and also $n-d-1$ edges in $T_i(r)$, we conclude that all
labels must be distinct. Further the labels are different from the ones seen at
$r$.

We now prove that $(G,{\cal T})$ satisfy the conditions in Proposition
\ref{prop:construction}. Essentially, we need to show that $|\hat{u}\cap
\hat{v}|=d$ implies that $uv$ is an edge in $G$.
Suppose $u,v$ are vertices in $G$
such that $|\hat{u}\cap \hat{v}|=d$. Assume $uv$ is not an edge of $G$. Let
$T_i$ be one of trees containing both $u$ and $v$. Let $w$ be an internal
vertex of the $u$-$v$ path in $T_i$.\smallskip

\noindent{\em Claim: } $\hat{u}\cap \hat{v}\subseteq \hat{u}\cap \hat{w}$. 

\noindent{\em Proof:}
If possible, let $j\in (\hat{u}\cap \hat{v})\backslash (\hat{u}\cap
\hat{w})$. Then $j\in \hat{u},\hat{v}$ but $x\not\in \hat{w}$. Hence, in
the oriented tree $T(w)$, there exist edges on paths $uv$ and $vw$ in
$T(w)$ with label $j$. But this contradicts the uniqueness of labels on the
edges of $T(w)$. Hence the claim.\smallskip

Let $u,z,w$ be the first three vertices on the $u$-$v$ path in the tree
$T_i$. Let $j$ and $k$ be the labels of edges $\ovr{uz}$ and $\ovr{zw}$
respectively. We show that $\{j,k\}\in \hat{u}\backslash \hat{w}$. Since
$j\not\in \hat{z}$, by the previous claim $j\not\in \hat{w}$. Also, we have
$k\not\in \hat{w}$. It remains to show that $k\in \hat{u}$. Suppose, on the
contrary that $k\not\in \hat{u}$. But then $k\in \hat{z}$, but $k\not\in
\hat{u},\hat{w}$. But then $d_G(z)\geq d_{T_k}(z)+2$, a contradiction to
the assumption (iii). Thus $\{j,k\}\subseteq \hat{u}\backslash \hat{w}$, or
$\hat{u}\cap \hat{v}\subseteq \hat{u}\cap \hat{w} = \hat{u}\backslash
(\hat{u}\cap \hat{w}) \subseteq \hat{u}\backslash \{j,k\}$. Thus
$|\hat{u}\cap \hat{v}|\leq d-1 < d$, a contradiction. This proves the
lemma.
\end{proof}

We are now in a position to prove Lemma \ref{lem:permselect}. 
\begin{proof}[Proof of Lemma $\ref{lem:permselect}$] 
We show that $(G,{\cal T})$ satisfy the conditions in Lemma
\ref{lem:construction2} for $d=4$.\smallskip

\noindent{\em Each tree has $n-4$ vertices}:
From Definition \ref{defn:tree}, the number of vertices in a tree is:
\begin{align*}
& 4(k+1) + (4k+1) & \quad [(i)+(ii)] \\
&+ 2k(0+1+2+3) & \quad [(iii) + (iv)] \\
&= 20k + 5 = n-4 = n-d & 
\end{align*}


\noindent{\em Any two trees intersect}: From Lemma \ref{lem:span}, it follows that any two trees in ${\cal D}T$
intersect.\smallskip

\noindent{\em Each vertex appears in $5$ trees, each edge in $4$-trees}:
Next, we calculate the number of trees that cover a particular
vertex $v\in V(G)$. Since $T_j=\varphi^j(T_0)$, we see that $|\{j: v\in
V(T_j)\}|=|\{j: \varphi^j(v)\in V(T_0)\}|$, which is same as the number of
vertices in $T_0$ from $\varphi$-orbit of $v$. Without loss of generality,
assume $v=v_{i,0}$ for some $0\leq i\leq 4k$. Clearly, when $v=v_{4i,0}$
for some $0\leq i\leq k$, $T_0$ contains $5$ vertices from $\varphi$-orbit
of $v$. Now let $v=v_{4i+t,0}$ where $1\leq t\leq 3$. Let $\varphi_v$
denote the $\varphi$-orbit of $v$. Note that $T_0$
intersects $\varphi_v$ at $v$. Other intersections between $T_0$ and
$\varphi_v$ occur along the downward and upward paths in $T_0$ from
vertices in $C_i$ and $C_{i+1}$ respectively. Now downward path of
length $\sigma_{i+1}(p)$ at a vertex in $C_i$ intersects $\varphi_v$ if $\sigma_{i+1}(p)\geq t$. Similarly an upward path
of length $\tau_{i+1}(q)$ at a vertex in $C_{i+1}$ intersects $\varphi_v$ if $\tau_{i+1}(q)\geq 4-t$. Recalling that
$\sigma_{i+1},\tau_{i+1}$ are permutations of $\{0,1,2,3\}$, and that the
respective paths are distinct, we get that $T_0$ contains $1+4-(4-t)+4-t=5$
vertices from $\varphi_v$. Similarly, it can be shown that $T_0$ contains
$4$ edges from each edge orbit under $\varphi$. As in the vertex case, this
implies that each edge is covered by exactly $4$ trees in ${\cal
T}$.\smallskip

\noindent{\em For a vertex $v\in V(T_i)$,
$d_G(v)-d_{T_i}(v)\leq 1$}:
Via the automorphism $\varphi$, it is sufficient to prove that for $v\in
V(T_0)$, $d_G(v)-d_{T_0}(v)\leq 1$. As $d_G(v_{i,j})=2$ for $i\not\cong 0$
(mod $4$), there is nothing to prove for those vertices. For remaining
vertices we have,
\begin{equation}
d_G(v_{4i,j}) = \begin{cases}
	3 & \text{if } i\in \{0,k\}, \\
	4 & \text{ otherwise. }
\end{cases}
\end{equation}
First consider the case when $i\in \{0,k\}$. Note that $v_{4i,j}\in V(T_0)$
for $j=0,d_i,\ldots,4d_i$. The vertices $v_{4i,0}$ have at least two
neighbors in $T_0$ (one on the path $P_0$, and other on the cycle $C_i$).
Thus the condition holds for these vertices. Similarly the vertices
$v_{4i,d_i},v_{4i,2d_i}$ and $v_{4i,3d_i}$ have degree at least two in $T_0$ (being internal
vertices of a path). The vertex $v_{4i,4d_i}$ has a neighbor $v_{4i,3d_i}$
on $T_0$, and an additional one on the downward (resp., upward) path at
$v_{4i,3d_i}$ when $i=0$ (resp., $k$). The preceeding claim holds because
$\sigma_1(3)\geq 1$ and $\tau_k(3)\geq 1$. Now consider the case when
$i\not\in \{0,k\}$. Then for $t=0,1,2$ the vertices $v_{4i,(t+1)d_i}$ are
internal vertices of a path in $T_0$. It has a further neighbor on an
upward or a downward path as $\sigma_i(t)+\tau_{i-1}(t)\geq 1$ for $t=0,1,2$.
Thus the vertices $v_{4i,(t+1)d_i}$ for $t=0,1,2$ have degree at least $3$ in $T_0$, and
hence they satisfy the condition. The vertices $v_{4i,4d_i}$ have a
neighbor $v_{4i,3d_i}$ and one neighbor each on upward and downward paths
at $v_{4i,4d_i}$ as the condition (c) implies both the paths are
non-trivial. 

Thus $(G,{\cal T})$ fufill the requirements of Lemma
\ref{lem:construction2}, and hence yeild tight neighborly four manifolds
with boundary. 
\end{proof}

\section{Implementation notes}
We describe an optimized algorithm to search for tight triangulations using
Lemma \ref{lem:construction2}. Given $k\geq
2$ and $n=20k+9$, our task is the following:
\begin{enumerate}[{\rm (i)}]
\item Find distinct invertible elements $d_0,\ldots,d_k$ in $\mathbb{Z}_n$.
Without loss of generality, we may choose $d_0=1$.
\item For a choice of $d_i:0\leq i\leq k$, search for $k$ pairs of
permutations
$\{(\sigma_i,\tau_i)\}_{i=1}^k$ of the set $\{0,1,2,3\}$ satisfying the
conditions in Lemma \ref{lem:permselect}.
\end{enumerate}

We explore the sequence $(d_0,\ldots,d_k)$ in a depth-first manner. To
prune the search tree, we use the following fact, which follows from the
proof of Lemma \ref{lem:tree}.
\begin{equation}\label{eq:prune}
\{\pm td_i:t=1,2,3,4\}\cap \{\pm td_j: t=1,2,3,4\} = \emptyset \text{ for }
i\neq j.
\end{equation}

Having determined a sequence $d_0,\ldots,d_k$, we look for a deck of $k$
permutation pairs $\{(\sigma_i,\tau_i)\}_{i=1}^k$. Again, we explore the
permutation pairs in a depth-first manner. The observations below help to
ecnomize the search.
From Lemma \ref{lem:permselect}, it follows that a valid permutation
$\sigma$ should satisfy $\sigma(t)=0$ for $t\in \{0,1,2\}$. Accordingly, we
call $\sigma$ to be of type $0$, $1$ or $2$ depending on whether
$\sigma(0)=0$, $\sigma(1)=0$ or $\sigma(2)=0$. Similarly, we call a
permutation pair $(\sigma,\tau)$ to be of type $(l,m)$ if $\sigma$ is of
type $l$ and $\tau$ is of type $m$. Since there are $6$ permutations of
each type, there are $6\times 6=36$ permutation pairs of each type. As
there are $9$ types of permuation pairs, we have $36\times 9=324$
permutations to consider at each level. However, we can substantially
reduce this number. We call permutation pairs of type
$(l,m)$ and $(l',m')$ to be {\em compatible} if $m\neq l'$. Observe that
the adjacent permutation pairs $(\sigma_i,\tau_i)$ and
$(\sigma_{i-1},\tau_{i-1})$ in a $k$-deck satisfying Lemma
\ref{lem:permselect} must be compatible. Thus, apart from the first level,
we only have to consider $216$ compatible permutation pairs. To enable
faster access to compatible permutations at each level, we do a
pre-processing step of storing them by their type. We store $36$
permutations of each type in a contiguous block. Then we stack $9$ such
blocks to form a linear array of $324$ permutation pairs. The blocks are
stacked following the lexicographic ordering of the type of the permutation
pairs they contain. It can be seen that in this scheme, all the permutation
pairs compatible with a given permutation pair, occur as contiguous blocks,
possibly wrapping around at the end of the array. 
Finally, we store a permutation pair $(\sigma,\tau)$ as the following set,
which we call its {\em treetype}.
\begin{equation}\label{eq:treetype}
treetype(\sigma,\tau) = \{(p+1,q+1): \sigma(p)+\tau(q)\geq 4\}.
\end{equation}
Given a set of six tuples $S$, there is at most one permutation pair
$(\sigma,\tau)$ such that $treetype(\sigma,\tau)=S$. The nomenclature
"treetype" denotes the fact that $(\sigma,\tau)$ determine the shape of
the tree at a particular level. 

\section{Results}
Following table summarizes the number of solutions for different values of
$k$. The $d$-vectors and treetypes of the solutions appear in the appendix.

\begin{table}[htbp]\label{tab:results}
\centering
\begin{tabular}{|c|c|c|c|}
\hline
$k$ & $n$ & \#(solutions) & Remarks \\
\hline
0 & 9 & 1 & K\"{u}hnel's twisted torus \\
\hline
1 & 29 & 6 & Includes examples in \cite{bdns} \\
\hline
2 & 49 & 1 & New \\
\hline
3 & 69 & 15 & '' \\
\hline
4 & 89 & 41 & '' \\
\hline
5 & 109 & 12 & '' \\
\hline
\end{tabular}
\caption{Examples of tight three manifolds}
\end{table}
%\bigskip



%{\footnotesize
{\small

\begin{thebibliography}{99}

\bibitem{BaTight}
B. Bagchi, 
A tightness criterion for homology manifolds with or without boundary,  {\em Euro. J. Combin.} (to appear), arXiv:1406.4299.

\bibitem{bd10} 
B. Bagchi, B. Datta, On Walkup's class ${\cal K}(d)$ and a minimal triangulation of $(S^{\hspace{.2mm}3} 
\mbox{$\times \hspace{-2.5mm}_{-}$} \, S^{\hspace{.1mm}1})^{\#3}$, {\em Discrete Math.} {\bf 311} (2011), 989--995.


\bibitem{bd16}
B. Bagchi, B. Datta, On stellated spheres and a tightness
criterion for combinatorial manifolds, {\em Euro. J. Combin.} {\bf 36}
(2014), 294--313.

\bibitem{bd17}
B. Bagchi, B. Datta, On $k$-stellated and $k$-stacked spheres, {\em Discrete
Math.} {\bf 313} (2013), 2318--2329.

\bibitem{bdns}
B. Datta, N. Singh, An infinite family of tight triangulations of manifolds,
{\em J. Combin. Theory (A)} {\bf 120} (2013), 2148--2163.

\bibitem{ef} F. Effenberger, Stacked polytopes and tight
triangulations of manifolds, {\em J. Combin. Theory} (A) {\bf 118}
(2011), 1843--1862.

\bibitem{simpcomp} F. Effenberger, J. Spreer, \emph{{\tt simpcomp}
-- a {\tt GAP} toolkit for simplicial complexes}, Version 1.5.4,
2011, http://www.igt.uni-stuttgart.de/LstDiffgeo/simpcomp.


\bibitem{ka} G. Kalai, Rigidity and the lower bound theorem 1, {\em
Invent.  math.} {\bf 88} (1987), 125--151.

\bibitem{NS} I. Novik, E. Swartz, Socles of Buchsbaum modules, complexes and
posets, {\em Adv. in Math.} {\bf 222} (2009), 2059--2084.

\bibitem{ku} W. K\"{u}hnel, {\em Tight Polyhedral Submanifolds and
Tight Triangulations}, Lecture Notes in Mathematics {\bf 1612},
Springer-Verlag, Berlin, 1995.

\bibitem{LSS} F. H. Lutz, T. Sulanke, E. Swartz, $f$-vector of
3-manifolds, {\em Electron. J. Comb.} {\bf 16} (2009), \#R\,13,
1--33.

\bibitem{si} N. Singh, Strongly minimal triangulations of
$(S^{\hspace{.2mm}3} \times S^{\hspace{.2mm}1})^{\#3}$ and
$(S^{\hspace{.2mm}3} \mbox{$\times \hspace{-2.5mm}_{-}$} \,
S^{\hspace{.1mm}1})^{\#3}$ (to appear in {\em Proc. Indian Academy of Sciences
(Math Sci.)}).

\bibitem{wa} D. W. Walkup, The lower bound conjecture for 3- and
4-manifolds, {\em Acta Math.} {\bf 125} (1970), 75--107.

\end{thebibliography} }

\end{document}



arXiv:1207.6182v1, 2012, 8 pages.
